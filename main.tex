\documentclass{statements}

\usepackage[margin=1in]{geometry}

\usepackage{amsthm}
\usepackage{amssymb}
\usepackage{amsmath}

\newcommand{\N}{\mathbb{N}}
\newcommand{\Z}{\mathbb{Z}}
\newcommand{\tand}{\text{ and }}

\title{
  NU1 Statements \\
  \large A Collection of Unsubstantiated Claims
}
\author{
  Nick Spinale \\
  \it Budapest Semesters in Mathematics \\
  \it Number Theory
}

\begin{document}

\maketitle

\section*{Divisibility}

\notation{
$a|b \iff \exists x : ax = b$
}

\prop{~
\begin{enum}
  \item $a|b \implies a|bc$
  \item $a|b \tand b|c \implies a|c$
  \item $a|b \tand a|c \implies a|bx+cy$
  \item $a|b \tand b \neq 0 \implies |a| \le |b|$
  \item $a|b \tand b|a \implies a = \pm b$
\end{enum}
}

\prop{
$a^n-b^n=(a-b)\sum^{n-1}_{i=0} a^{n-i}b^i$
}

\prop{
If $n$ is odd, then
$a^n+b^n=(a+b)\sum^{n-1}_{i=0} (-1)^ia^{n-i}b^i$
}

\prop{
If $n$ is composite, then $2^n-1$ is also composite.
}

\prop{
If $n\ge 2$ and $a^n-1$ is prime, then $a=2$ and $n$ is prime.
}

\dfn[Fermat numbers]{
$F_n=2^{(2^n)}+1$
}

\prop{
$(F_n,F_m)=1$
}

\dfn{
$d$, denoted $(a,b)$, is the distingiushed common divisor of $a$ and $b$ iff
\begin{enum}
  \item $d|a \tand d|b$
  \item $c|a \tand c|b \implies c|d$
\end{enum}
}

\prop{
$(a, b)$ exists, and is unique up to sign.
}

\dfn[Euclidean Algorithm]{
Todo
}

\prop{~
\begin{enum}
  \item $(a,b)=(a,ak+b)$
  \item $(ma,mb)=m(a,b)$
\end{enum}
}

\dfn[Euclidean Algorithm]{
$(a,b)$ is the smallest $n$ such that $ax+by=n$.
}

\dfn{
$a$ and $b$ are relatively prime iff $(a,b)=1$.
}

\lemma[Euclid]{
$a|bc \tand (a,b)=1 \implies a|c$
}

\section*{Base 10 Divisibility}

\prop[Divisibility by 9]{
$\overline{a_k\ldots a_1a_0} \equiv a_k+\ldots+a_1+a_0 \mod 9$
}

\prop[Divisibility by 11]{
$\overline{a_k\ldots a_1a_0} \equiv \sum^k_{i=0} (-1)^{n}a_i \mod 11$
}

\prop[Last $k$ digit rule]{
If $n|10^k$, then $\overline{\ldots a_ka_{k-1} \ldots a_1a_0} \equiv \overline{a_{k-1} \ldots a_2 a_1} \mod n$
}

\section*{Primes}

\dfn{
$p$ is irreducable iff $a|p \implies a=1 \lor a=p$
}

\dfn{
$p$ is prime iff $p|ab \implies p|a \lor p|b$
}

\prop{
In $\Z$, irreducability and primality are equivalent.
}

\thm[Fundimental Theorem of Arithmetic]{
Every positive integer $n$ has a unique canonical representation
$$n = p_1^{\alpha_1} p_2^{\alpha_2} \ldots p_k^{\alpha_k} = \prod^k_{i=1} p_i^{\alpha_i}$$
Where $p_1 < p_2 < \ldots < p_k$ are primes.
}

\thm[Bertrand's Postulate]{
For all $n$, there exists a prime $p$ such that $n < p < n2$.
}

\thm{
Arbitrarily large prime gaps exist.
}

\thm[Dirichet]{
If $(a,b)=1$, then there are infinitely many primes of the form $ak+b$.
}

\thm{
$(\exists x: x^2 \equiv -1 \mod p) \iff p=4k+1$
}

\thm{
If $p=4k-1$ and $p|a^2+b^2$, then $p|a$, $p|b$, and $p^2|a^2+b^2$.
}

\section*{Congruences}

\prop{
Assume $a\equiv b\mod m$ and $c\equiv d\mod m$.
\begin{enum}
  \item $a+c \equiv b+d$
  \item $ac \equiv bc$
  \item $ac \equiv bd$
  \item $a^n \equiv b^n$
\end{enum}
}

\dfn{~
\begin{enum}
  \item A complete residue system modulo $n$ is a set containing exactly one element from each residue class modulo $n$.
  \item A reduced residue system modulo $n$ is a set containing exactly one element from each residue class modulo $n$ coprime to $m$.
\end{enum}
}

\dfn[Totient function]{
$\phi(n)$ is the number of integers $a$ such that $1 \le a < n$ such that $(a,n)=1$.
}

\prop{
Assume $n = p_1^{\alpha_1}p_2^{\alpha_2}\ldots p_k^{\alpha_k}$.
Then $\phi(n) = \prod (p_i^{\alpha_i}-p_i^{\alpha_i-1}) = n\prod(1-\frac{1}{p_i})$
}

\thm[Euler's totient theorem]{
If $(a,n)=1$, then $a^{\phi(n)}\equiv 1 \mod n$.
}

\thm[Wilson]{
$(p-1)! \equiv -1 \mod p$
}

\thm{
The congruence $ax\equiv b \mod m$ has a solution iff $(m,a)|b$.
Furthermore, all such solutions are equivalent modulo $\frac{m}{(m,a)}$.
}

\cor{
$ab\equiv ac\mod m \iff b\equiv c\mod\frac{m}{(m,a)}$
}

\cor{
The equation $ax+by=c$ has a solution iff $(a,b)|c$.
Furthermore, if $(x_0,y_0)$ is a solution, then so is $(x_0-t\frac{b}{(a,b)}, y_0+t\frac{a}{(a,b)})$.
}

\newcommand{\zmod}[1]{\ensuremath{\mathbb{Z}/#1\mathbb{Z}}}
\thm[Chinese Remainder Theorem]{
If $n_1, \dots n_k$ are pairwise coprime and $\Pi n_i = N$,
then $x \mod N \mapsto (x \mod n_1, \dots x \mod n_k)$ is a ring isomorphism.
}

\section*{Interesting Numbers}

\thm{
If $m$ and $n$ are each the sum of two squares, then so is $mn$.
}

\thm{
$n$ is the sum of two squares iff $n = 2^\gamma \prod p_i^{\alpha_i} \prod q_i^{2\beta_i}$,
where $p_i=4k+1$ and $q_i=4k-1$.
}

\section*{Order}

\dfn[Order]{
Given modulus $n$ and $g$ such that $(n,a) = 1$,
the order of $g$ is the smallest positive $k$ such that $g^k \equiv 1 (n)$.
We say $o_n(g) = k$.
}

\prop{
$o_n(g) | \phi(n)$
}

\prop{
$o_n(g^i) = \frac{o_n(g)}{(i,o_n(g))}$
}

\dfn[Primitive Root]{
$g$ is a primitive root modulo $n$ if $o_n(g) = \phi(n)$.
}

\thm{
There exists a primitive root modulo $n$ iff one of the following is true:
\begin{align*}
  n &= p^\alpha \text{ for some odd prime } p \\
  n &= 2p^\alpha \text{ for some odd prime } p \\
  n &= 2 \\
  n &= 4 \\
\end{align*}
}

\thm{
$\sum_{d|n} \phi(n) = n$
}

\section*{Quadratic Residues}

\end{document}

% TODO
% - sum of squares
