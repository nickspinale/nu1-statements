\documentclass{statements}

\usepackage[margin=1in]{geometry}

\usepackage{amsthm}
\usepackage{amssymb}
\usepackage{amsmath}

\newcommand{\N}{\mathbb{N}}
\newcommand{\Z}{\mathbb{Z}}
\newcommand{\tand}{\text{ and }}

\title{
  NU1 Statements \\
  \large A Collection of Unsubstantiated Claims
}
\author{
  Nick Spinale \\
  \it Budapest Semesters in Mathematics \\
  \it Number Theory
}

\begin{document}

\maketitle

\section*{Divisibility}

\notation{
$a|b \iff \exists x : ax = b$
}

\prop{
-
\begin{enum}
  \item $a|b \implies a|bc$
  \item $a|b \tand b|c \implies a|c$
  \item $a|b \tand a|c \implies a|bx+cy$
  \item $a|b \tand b \neq 0 \implies |a| \le |b|$
  \item $a|b \tand b|a \implies a = \pm b$
\end{enum}
}

\prop{
$a^n-b^n=(a-b)\sum^n_{i=0} a^{n-i}b^i$
}

\prop{
If $n$ is odd, then
$a^n+b^n=(a+b)\sum^n_{i=0} (-1)^ia^{n-i}b^i$
}

\dfn{
$d$, denoted $(a,b)$, is the distingiushed common divisor of $a$ and $b$ iff
\begin{enum}
  \item $d|a \tand d|b$
  \item $c|a \tand c|b \implies c|d$
\end{enum}
}

\prop{
$(a, b)$ exists, and is unique up to sign.
}

\dfn[Euclidean Algorithm]{
Todo
}

\prop{
-
\begin{enum}
  \item $(a,b)=(a,ak+b)$
  \item $(a,b)=(ma,mb)$
\end{enum}
}

\dfn{
$a$ and $b$ are relatively prime iff $(a,b)=1$.
}

\lemma[Euclid]{
$a|bc \tand (a,b)=1 \implies a|c$
}

\section*{Base 10 Divisibility}

\prop[Divisibility by 9]{
$\overline{a_k\ldots a_2a_1} \equiv a_k+\ldots+a_2+a_1 \mod 9$
}

\prop[Divisibility by 11]{
$\overline{a_k\ldots a_2a_1} \equiv \sum^k_{i=1} (-1)^{n-1}a_i \mod 11$
}

\prop[Last $n$ digit rule]{
If $a|10^k$, then $\overline{\ldots a_k \ldots a_2 a_1} \equiv \overline{a_k \ldots a_2 a_1} \mod a$
}

\section*{Primes}

\dfn{
$p$ is irreducable iff $a|p \implies a=1 \lor a=p$
}

\dfn{
$p$ is prime iff $p|ab \implies p=a \lor p=b$
}

\thm{
In $\Z$, irreducability and primality are equivalent.
}

\thm[Fundimental Theorem of Arithmetic]{
Every positive integer $n$ has a unique canonical representation
$$n = p_1^{\alpha_1} p_2^{\alpha_2} \ldots p_k^{\alpha_k} = \prod^k_{i=1} p_i^{\alpha_i}$$
Where $p_1 < p_2 < \ldots < p_k$ are primes.
}

\thm{
There are infinitely many primes
}

\section*{Congruences}

\end{document}
